% Copyright 2023 Yoshida Shin
%
% This is part of the ``WEBASSEMBLY What's the right thing to write?''.
%
% Permission is granted to copy, distribute and/or modify this
% document under the terms of the GNU Free Documentation License,
% Version 1.3 or any later version published by the Free Software
% Foundation; with one Invariant Sections:
% ``Shin Yoshida wrote this document with the goal of contributing to a fair and safe world.
% Funai Soken Digital Incorporated agrees with the vision and compensated him for his work.''
% no Front-Cover Texts, and no Back-Cover Text. A copy of the license is
% included in the section entitled ``GNU Free Documentation License''.

\section{The Second Generation Computer}


\begin{frame}{}{}
    {\Large The Second Generation Computer}
\end{frame}


\begin{frame}{The Second Generation Computer}{}
    The second generation computers
    \vspace{2ex}

    \onslide+<2->{They were developed using semiconductor instead of vacuum tube.}
    \vspace{4ex}

    \onslide+<3->{Most current computers are second generation.}
\end{frame}


\begin{frame}{The Second Generation Computer}{}
    Semiconductor devices are much lighter than vacuum tube.
    \vspace{4ex}

    \onslide+<2->{The second generation computer is much lighter than the firsts.}
    \vspace{4ex}

    \onslide+<3->{The specialized building is not necessary.}
\end{frame}


\begin{frame}{The Second Generation Computer}{}
    Semiconductor eliminated the need for a specialized building.
\end{frame}


\begin{frame}{ENIAC}{The Second Generation Computer}
    btw,
    \vspace{2ex}

    ENIAC extracted the logic from the circuit.
    \vspace{4ex}

    \onslide+<2->{Then, don't you like to create specification?}
\end{frame}


\begin{frame}{After ENIAC}{The Second Generation Computer}
    This is an example of the work flow after ENIAC.
    \vspace{4ex}

    \begin{enumerate}
        \onslide+<2->{\item Decide the hardware specification.}
        \onslide+<3->{\item Create a computer (hardware) complying with the specification.}
        \onslide+<4->{\item Create a program complying with the specification.}
    \end{enumerate}
\end{frame}


\begin{frame}{After ENIAC}{The Second Generation Computer}
    Then, the programmer don't have to study the circuit.
    \vspace{4ex}

    \onslide+<2->{This allows more compatibility between different programs and hardware.}
\end{frame}


\begin{frame}{After ENIAC}{The Second Generation Computer}
    The specification make the hardware abstract.
    \vspace{4ex}

    \onslide+<2->{{\footnotesize ``x86\_64'' is a famous specification of CPU.}}
\end{frame}