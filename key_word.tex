% Copyright 2023 Yoshida Shin
%
% This is part of the ``WEBASSEMBLY What's the right thing to write?''.
%
% Permission is granted to copy, distribute and/or modify this
% document under the terms of the GNU Free Documentation License,
% Version 1.3 or any later version published by the Free Software
% Foundation; with one Invariant Sections:
% ``Shin Yoshida wrote this document with the goal of contributing to a fair and safe world.
% Funai Soken Digital Incorporated agrees with the vision and compensated him for his work.''
% no Front-Cover Texts, and no Back-Cover Text. A copy of the license is
% included in the section entitled ``GNU Free Documentation License''.

\section{Key Word}


\begin{frame}{}{}
    {\Huge Key Word}
\end{frame}


\begin{frame}{Key Word}{}
    The keywords of this document are ``split'' and ``abstraction''.
    \vspace{4ex}

    Let me briefly explain them.
\end{frame}


\begin{frame}{Key Word}{}
    {\Huge Split}
\end{frame}


\begin{frame}{Split}{Key Word}
    A large work can be overwhelming.
    \vspace{4ex}

    \begin{itemize}
        \onslide+<2->{\item Probably it includes many difficult points.}
        \onslide+<3->{\item Probably there are many uncertainties.}
    \end{itemize}
    \vspace{4ex}

    \onslide+<4->{These factors can discourage the person assigned to the task.}
    \vspace{4ex}

    \onslide+<5->{It is difficult to deal with many things at once alone.}
\end{frame}


\begin{frame}{Split}{Key Word}
    {\Large That's where splitting comes in!}
\end{frame}


\begin{frame}{Split}{Key Word}
    By splitting a large task into smaller components,
    \vspace{4ex}

    \begin{itemize}
        \onslide+<2->{\item Multiple people can work together.
            \begin{itemize}
                \onslide+<3->{\item Each person can focus on their specialized field.}
                \onslide+<4->{\item They can give advice each other.}
            \end{itemize}}
        \onslide+<5->{\item The assignee can then concentrate on a smaller part, reducing the amount of things they need to think about simultaneously.}
        \onslide+<6->{\item Splitting a task also make progress more visible.}
    \end{itemize}
\end{frame}


\begin{frame}{Split}{Key Word}
    There are some disadvantage about splitting, of course.
    \vspace{4ex}

    \begin{itemize}
        \onslide+<2->{\item Splitting itself is a task.}
        \onslide+<3->{\item We have to manage the small tasks. It is a task as well.}
    \end{itemize}
\end{frame}


\begin{frame}{Key Word}{}
    {\Huge Abstraction}
\end{frame}


\begin{frame}{Abstraction}{Key Word}
    Abstraction is the process of reducing a complex system or concept to its essential features.
    \vspace{4ex}

    \onslide+<2->{In the context of computer science, it is often about splitting and clarify external specifications or creating standards.}
    \vspace{2ex}

    \onslide+<3->{Both specificsation and the standard are similar to user's manual.}
\end{frame}


\begin{frame}[t]{Abstraction}{Key Word}
    For example, let's say your company want to develop 10 SD cards and 4 smartphones to use the SD cards.
    \vspace{4ex}

    \begin{itemize}
        \onslide+<2->{\item[Cost]   Developping (10 + 4) new products.}
        \onslide+<3->{\item[Usage]    Selecting from (10 x 4) pairs.}
    \end{itemize}
    \vspace{4ex}

    \onslide+<4->{Getting a lot of results consuming less resources.}
\end{frame}


\begin{frame}{Abstraction}{Key Word}
    The standard of SD card abstracts the storage.
    \vspace{4ex}

    \onslide+<2->{The SD cards complies with the standard.}
    \vspace{2ex}

    \onslide+<3->{(The SD cards behaves as written in the manual.)}
    \vspace{4ex}

    \onslide+<4->{The smartphones complies with the standard.}
    \vspace{2ex}

    \onslide+<5->{(The smartphones uses the SD cards as written in the manual.)}
\end{frame}


\begin{frame}{Abstraction}{Key Word}
    ``Getting a lot of results consuming less resources.''
    \vspace{4ex}

    \onslide+<2->{This is one of the vartue of abstraction.}
    \vspace{4ex}

    \onslide+<3->{But the example shows some other advantages of abstraction.}
\end{frame}


\begin{frame}{Abstraction}{Key Word}
    Each developer only needs to understand the device they are in charge of.
    \vspace{4ex}

    \onslide+<2->{The standard decouples the smartphone and the SD card more.}
\end{frame}


\begin{frame}{Abstraction}{Key Word}
    The standard guarantees the minimum quality of the SD cards.
    \vspace{4ex}

    \onslide+<2->{Your company may be new to SD card, but the products fulfill the minimum quality as long as they complies with the standard.}
\end{frame}


\begin{frame}{Abstraction}{Key Word}
    The standard make the task dependent on an individual person.
    \vspce{4ex}

    It will reduce the documentation and user's manual as well.
\end{frame}


\begin{frame}{Abstraction}{Key Word}
    Discussion
    \vspace{4ex}

    What is another advantage of abstraction?
    \vspace{2ex}

    What is the difference between specification and standard in business?
\end{frame}


\begin{frame}{Key Word}
    Split and abstraction reduces the effort.
    \vspace{4ex}

    \onslide+<2->{This is what a lazy IT engineer loves!}
    \vspace{2ex}

    \onslide+<3->{{\Large (Lazy: One of the best virtue)}}
\end{frame}


\begin{frame}{Key Word}{}
    Lazy engineers are often the ones who come up with the most efficient solutions.
    \vspace{4ex}

    \onslide+<2->{They don't want to waste time and energy doing things that don't matter.}
    \vspace{4ex}

    \onslide+<3->{Instead, they focus on finding ways to reduce the effort required to achieve their goals.}
\end{frame}


\begin{frame}{Key Word}{}
    This is a valuable trait in the world of computer science, where efficiency is highly prized.
    \vspace{4ex}

    \onslide+<2->{And WebAssembly is created.}
\end{frame}


\begin{frame}{Key Word}{}
    Discussion
    \vspace{4ex}

    What is disdvantage of splitting and abstraction?
\end{frame}
