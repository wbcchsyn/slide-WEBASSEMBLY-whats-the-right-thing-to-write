% Copyright 2023 Yoshida Shin
%
% This is part of the ``WEBASSEMBLY What's the right thing to write?''.
%
% Permission is granted to copy, distribute and/or modify this
% document under the terms of the GNU Free Documentation License,
% Version 1.3 or any later version published by the Free Software
% Foundation; with one Invariant Sections:
% ``Shin Yoshida wrote this document with the goal of contributing to a fair and safe world.
% Funai Soken Digital Incorporated agrees with the vision and compensated him for his work.''
% no Front-Cover Texts, and no Back-Cover Text. A copy of the license is
% included in the section entitled ``GNU Free Documentation License''.

\section{WebAssembly System Interface}


\begin{frame}{}{}
    {\Large WebAssembly System Interface}
\end{frame}


\begin{frame}{WebAssembly System Interface}{}
    Remember the sentence in section ``WebAssembly.''
    \vspace{4ex}

    \onslide+<2->{\begin{quote}Give up running everything on VM.\end{quote}}
\end{frame}


\begin{frame}{WebAssembly System Interface}{}
    Some people did not like this idea.
    \vspace{4ex}

    \onslide+<2->{They created the standard of the interface that allows WebAssembly to interact with the OS}
    \vspace{4ex}

    \onslide+<3->{This is known as ``WebAssembly System Interface,'' or WASI for short.}
\end{frame}


\begin{frame}[t]{WebAssembly System Interface}{}
    Remember the example of the average application.
    \vspace{2ex}
    
    \onslide*<2>{
        \begin{enumerate}
            \item OS kicks off the application written in JavaScript.
            \item JavaScript kicks off WebAssembly VM.
            \item JavaScript passes the functions for Input/Output to the VM.
            \item The VM retrives the numbers using the passed function.
            \item The VM calculates the average.
            \item The VM displays the answer using the passed function.
        \end{enumerate}
    }
    \onslide*<3->{
        \begin{enumerate}
            \item ...
            \item JavaScript kicks off WebAssembly VM.
            \item JavaScript passes the functions for Input/Output to the VM.
            \item ...
        \end{enumerate}
        \vspace{4ex}

        \onslide+<4->{Can we prepare these functions in advance
        so that the WebAssembly VM can send all requests to the OS?}
    }
\end{frame}


\begin{frame}{WebAssembly System Interface}{}
    {\Large Yes we can.}
    \vspace{4ex}

    \onslide+<2->{That's exactly what WASI defines:}
    \vspace{2ex}
    
    \onslide+<3->{a set of functions to be passed to the WebAssembly VM so it can communicate directly with the OS and hardware.} 
\end{frame}


\begin{frame}{WebAssembly System Interface}{}
    WASI is a new standard that defines these functions to be passed to the WebAssembly VM.
    \vspace{4ex}
    
    \onslide+<2->{Although many changes are occurring in a short period of time,
    some people are already planning to use a new eco-system using WASI.}
\end{frame}


\begin{frame}{WebAssembly System Interface}{}
    For example we can incorporate WebAssembly into Linux OS.
    \vspace{4ex}

    \onslide+<2->{This will reduce the overhead of the VM.}
\end{frame}


\begin{frame}{WebAssembly System Interface}{}
    Alternatively, we can create a new OS on the WASI
    \vspace{4ex}
    
    \onslide+<2->{users wouldn't have to worry about the underlying OS and hardware.}
\end{frame}


\begin{frame}{WebAssembly System Interface}{}
    Discussion
    \vspace{4ex}

    How do we use WebAssembly System Interface in business?
\end{frame}