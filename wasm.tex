% Copyright 2023 Yoshida Shin
%
% This is part of the ``WEBASSEMBLY What's the right thing to write?''.
%
% Permission is granted to copy, distribute and/or modify this
% document under the terms of the GNU Free Documentation License,
% Version 1.3 or any later version published by the Free Software
% Foundation; with one Invariant Sections:
% ``Shin Yoshida wrote this document with the goal of contributing to a fair and safe world.
% Funai Soken Digital Incorporated agrees with the vision and compensated him for his work.''
% no Front-Cover Texts, and no Back-Cover Text. A copy of the license is
% included in the section entitled ``GNU Free Documentation License''.

\section{WebAssembly}


\begin{frame}{}{}
    {\Huge WebAssembly}
\end{frame}


\begin{frame}{WebAssembly}{}
    It was a great idea to run an application on Virtual Machine.
    \vspace{4ex}

    \onslide+<2->{However, Java left 2 things to be improved in retrospect.}
    \vspace{4ex}

    \begin{itemize}
        \onslide+<3->{\item ``Java technology'' is too large.}
        \onslide+<4->{\item ``Write once, run anywhere'', it is impossible.}
    \end{itemize}
\end{frame}


\begin{frame}{Java is too large}{WebAssembly}
    ``Java technology'' includes many things.
    \vspace{4ex}

    \begin{itemize}
        \onslide+<2->{\item Java language}
        \onslide+<3->{\item Java compiler}
        \onslide+<4->{\item Java application}
        \onslide+<5->{\item Java virtual machine}
        \onslide+<6->{\item ...}
    \end{itemize}
\end{frame}


\begin{frame}{Java is too large}{WebAssembly}
    What if we separated and abstracted these components?
\end{frame}


\begin{frame}{Java is too large}{WebAssembly}
    Why do we have to write code in the specific language to use the VM?
    \vspace{4ex}

    \onslide+<2->{VM is one thing and the language is quite another.}
    \vspace{4ex}

    \onslide+<3->{We need only a standard to abstract this.}
\end{frame}


\begin{frame}{WebAssembly}{}
    {\Large WebAssembly is a starndard to abstract the VM.}
\end{frame}


\begin{frame}{WebAssembly}{}
    Thanks to WebAssembly, we can create multipile VMs and compilers.
    \vspace{4ex}

    \begin{itemize}
        \onslide+<2->{\item Programmers can write source code in their favorite language (as long as the compiler is available.)}
        \onslide+<3->{\item Users can select one compiler and one VM form many implementations.}
    \end{itemize}
\end{frame}


\begin{frame}{WebAssembly}{}
    The behavior of WebAssembly applications will be consistent anywhere.
\end{frame}


\begin{frame}{WebAssembly}{}
    {\Huge Really?}
    \vspace{4ex}

    \onslide+<2->{Could you print out this slide using a computer without any printer?}
\end{frame}


\begin{frame}{WebAssembly}{}
    ``Write once, run anywhere''
    \vspace{4ex}

    It is impossible to create such a program.
\end{frame}


\begin{frame}{WebAssembly}{}
    What should we do?
\end{frame}


\begin{frame}{WebAssembly}{}
    Goal
    \vspace{2ex}

    WebAssembly should work well anywhere.
    \vspace{4ex}

    \onslide+<2->{
        Problem
        \vspace{2ex}

        Any program (including a VM) is dependent on the OS and hardware.
    }
\end{frame}


\begin{frame}{WebAssembly}{}
    Remember
    \vspace{2ex}

    \onslide+<2->{An application performs two kinds of processes.}
    \vspace{4ex}

    \begin{itemize}
        \onslide+<3->{\item Request to the OS}
        \onslide+<4->{\item Calculation using the CPU and memory directly}
    \end{itemize}
\end{frame}


\begin{frame}{WebAssembly}{}
    Give up running everything on VM.
    \vspace{4ex}

    \onslide+<2->{WebAssembly should treat only what is independent of the OS.}
    \vspace{4ex}

    \onslide+<3->{i.e. WebAssembly should not send a request to the OS.}
\end{frame}


\begin{frame}{Average}{WebAssembly}
    Let's consider to create an application to calculate the average.
    \vspace{4ex}

    \onslide+<2->{Usage}
    \vspace{2ex}

    \begin{enumerate}
        \onslide+<3->{\item The user enters some numbers}
        \onslide+<4->{\item The application displays the average}
    \end{enumerate}
\end{frame}


\begin{frame}{WebAssembly}{}
    Imagine to create the program JavaScript and WebAssembly.
\end{frame}


\begin{frame}{WebAssembly}{}
    Here is an example of the process flow.
    \vspace{2ex}

    \begin{enumerate}
        \onslide+<2->{\item OS kicks off the application written in JavaScript.}
        \onslide+<3->{\item JavaScript retrives the numbers from the keyboard.}
        \onslide+<4->{\item JavaScript kicks off the WebAssembly VM.}
        \onslide+<5->{\item JavaScript passes the numbers to the VM.}
        \onslide+<6->{\item The VM calculates the average.}
        \onslide+<7->{\item The VM passes the result to JavaScript.}
        \onslide+<8->{\item JavaScript displays the answer.}
    \end{enumerate}
\end{frame}


\begin{frame}{WebAssembly}{}
    Here is another example
    \vspace{2ex}

    \begin{enumerate}
        \onslide+<2->{\item OS kicks off the application written in JavaScript.}
        \onslide+<3->{\item JavaScript kicks off WebAssembly VM.}
        \onslide+<4->{\item JavaScript passes the functions for Input/Output to the VM.}
        \onslide+<5->{\item The VM retrives the numbers using the passed function.}
        \onslide+<6->{\item The VM calculates the average.}
        \onslide+<7->{\item The VM displays the answer using the passed function.}
    \end{enumerate}
\end{frame}


\begin{frame}{WebAssembly}{}
    You can use WebAssembly partially to write the source code.
    \vspace{4ex}

    \onslide+<2->{The code for WebAssembly will work anywhere.}
    \vspace{4ex}

    \onslide+<3->{You must write JavaScript code anyway. It may depends on the computer.}
\end{frame}


\begin{frame}{WebAssembly}{}
    {\Large How does WebAssembly used in the business?}
\end{frame}


\begin{frame}{WebAssembly}{}
    Ex1) Enhancing JavaScript on WebPage
    \vspace{4ex}

    \begin{itemize}
        \onslide+<2->{\item Web browsers can only run JavaScript.}
        \onslide+<3->{\item Sometimes it is not easy to write a complicated code in JavaScript.}
        \onslide+<4->{\item WebAssembly usually performs better than JavaScript.}
    \end{itemize}
    \vspace{4ex}

    \onslide+<5->{WebAssembly may make it easy to create a good program.}
\end{frame}


\begin{frame}{WebAssembly}{}
    Ex2) Framework utilizing WebAssembly VM
    \vspace{4ex}

    \onslide+<2->{In case of Ruby on Rails, for example, you must write the code in Ruby.}
    \vspace{4ex}

    \onslide+<3->{However, with a framework that utilizes the WebAssembly VM, code can be written in any language.}
\end{frame}


\begin{frame}{WebAssembly}{}
    Remember the introduction again.
\end{frame}


\begin{frame}{WebAssembly}{}
    According to webassembly.org,
    \begin{quote}WebAssembly (abbreviated Wasm) is a binary instruction format for a stack-based virtual machine.\end{quote}
    \vspace{2ex}

    \onslide+<2->{
        \begin{itemize}
            \item   instraction:    The specification of CPU.
            \item   stack-based:    A kind of computer architecture.
        \end{itemize}
    }
    \vspace{2ex}

    \onslide+<3->{I think the sentence means that WebAssembly is a standard to make the programming logic abstract.}
\end{frame}
