% Copyright 2023 Yoshida Shin
%
% This is part of the ``WEBASSEMBLY What's the right thing to write?''.
%
% Permission is granted to copy, distribute and/or modify this
% document under the terms of the GNU Free Documentation License,
% Version 1.3 or any later version published by the Free Software
% Foundation; with one Invariant Sections:
% ``Shin Yoshida wrote this document with the goal of contributing to a fair and safe world.
% Funai Soken Digital Incorporated agrees with the vision and compensated him for his work.''
% no Front-Cover Texts, and no Back-Cover Text. A copy of the license is
% included in the section entitled ``GNU Free Documentation License''.

\section{Java}

\begin{frame}{}{}
    {\Huge Java}
\end{frame}


\begin{frame}{Java}{}
    The development of ``Unix and C'' made it easier to create applications.
    \vspace{4ex}

    \onslide+<2->{However, it also led to new problems.}
    \vspace{4ex}

    \onslide+<3->{For example, the same program may not work on different OS even if the hardware is the same.}
\end{frame}


\begin{frame}{Java}{}
    Java created ``Java Virtual Machine (abbreviated JVM)'' and tried to solve this problem.
    \vspace{4ex}

    \onslide+<2->{``Write once, run anywhere'' is a chatchword of Java.}
\end{frame}


\begin{frame}{Java Virtual Machine}{Java}
    What is Virtual Machine?
\end{frame}


\begin{frame}{Java Virtual Machine}{Java}
    According to Cambridge Dictionary, ``Virtual'' means
    \vspace{4ex}

    \begin{itemize}
        \onslide+<2->{\item almost a particular thing or quality

            {\footnotesize Ten years of incompetent government had brought about the virtual collapse of the country's economy.}}
        \onslide+<3->{\item created by computer technology and appearing to exist but not existing in the physical world

            {\footnotesize In the game players simulate real life in a virtual world.}}
    \end{itemize}
\end{frame}


\begin{frame}{Java Virtual Machine}{Java}
    In IT space, I think ``Virtual'' stands for 
    \vspace{4ex}

    \onslide+<2->{``having the all essential functions and features.''}
    \vspace{4ex}

    \onslide+<3->{It often indicates ``not limited by the extrinsic features.''}
\end{frame}


\begin{frame}{Java Virtual Machine}{Java}
    What is machine?
    \vspace{4ex}

    \onslide+<2->{IT engineers often call a computer ``Machine''.}
    \vspace{4ex}

    \onslide+<3->{Here, ``Machine'' means where an application runs.}
\end{frame}


\begin{frame}{Java Virtual Machine}{Java}
    JVM is a program to run a Java application.
    \vspace{4ex}

    \onslide+<2->{Java application: application complying with the specification of JVM.}
    \vspace{4ex}

    \onslide+<3->{(JVM has a function to run a Java application, and not limited by the power cable.)}
\end{frame}


\begin{frame}{Java}{}
    How can we write code complying with the specification of JVM?
    \vspace{4ex}

    \onslide+<2->{Write source code in Java language.}
    \vspace{4ex}

    \onslide+<3->{Java compiler generates a binary formatted file that JVM can execute.}
\end{frame}


\begin{frame}{Java}{}
    i.e. Java language abstracts the JVM.
    \vspace{4ex}

    JVM abstracts the OS.
\end{frame}


\begin{frame}{Java}{}
    Here is an example of the work flow.
    \vspace{4ex}

    \begin{enumerate}
        \onslide+<2->{\item Oracle creates Java Virtual Machine for the OS and the hardware.}
        \onslide+<3->{\item Software engineer write source code in Java language.}
        \onslide+<4->{\item Java compiler generates an Java application.}
    \end{enumerate}
\end{frame}


\begin{frame}{Caution}{Java}
    The meaning of ``Machine'' is ambiguous.
    \vspace{2ex}

    \onslide+<2->{It often referes to ``Something where OS runs''.}
    \vspace{4ex}

    \onslide+<3->{Therefore, ``Virtual Machine'' may have another meaning.}
\end{frame}
