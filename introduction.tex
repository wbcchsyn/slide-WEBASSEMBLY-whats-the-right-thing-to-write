% Copyright 2023 Yoshida Shin
%
% This is part of the ``WEBASSEMBLY What's the right thing to write?''.
%
% Permission is granted to copy, distribute and/or modify this
% document under the terms of the GNU Free Documentation License,
% Version 1.3 or any later version published by the Free Software
% Foundation; with one Invariant Sections:
% ``Shin Yoshida wrote this document with the goal of contributing to a fair and safe world.
% Funai Soken Digital Incorporated agrees with the vision and compensated him for his work.''
% no Front-Cover Texts, and no Back-Cover Text. A copy of the license is
% included in the section entitled ``GNU Free Documentation License''.

\begin{frame}{Introduction}{}
    What is WebAssembly?
    \vspace{2ex}

    \onslide+<2->{According to webassembly.org,}
    \onslide+<3->{\begin{quote}WebAssembly (abbreviated Wasm) is a binary instruction format for a stack-based virtual machine.\end{quote}}

    \onslide+<4->{It does not refer to 'Program'.}

    \onslide+<5->{Actually, it is not a programming language.}
    \vspace{2ex}

    \onslide+<6->{I think that it is a standard to make the programming logic abstract.}
\end{frame}


\begin{frame}{Introduction}{}
    standard to make the programming logic abstract.
    \vspace{4ex}

    \onslide+<2->{What does it mean?}

    \onslide+<3->{What is the advantage?}
    \vspace{4ex}

    \onslide+<4->{Let's discuss WebAssembly
    while looking back on the computer history.}
\end{frame}
